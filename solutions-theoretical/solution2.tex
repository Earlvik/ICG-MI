% header
\documentclass[10pt,a4paper]{article}

\usepackage[latin1]{inputenc}
\usepackage{hyperref}
\usepackage{amssymb}
\usepackage{amsmath}
\usepackage{ngerman}
\usepackage{graphicx}

% the document
\begin{document}

% create the title
% Please replace the data in brackets [] with actual data.
\title{Abgabe - �bungsblatt 2\\
\small{Einf�hrung in die Computergraphik und Visualisierung}}
\author{ Svetlana Shishkovets \and Tran Linn Chi \and Viktor Lopatin}
\date{\today}
\maketitle

\section*{Exercise 1}
Point $p$ can be embedded in a quaternion as \\
$p \mapsto q_p = (0, p) = xi + yj + zk$. \\
\medskip

\textbf{(a)} Give a formula to determine the angle of rotation $\alpha$.\\
The angle of rotation is an angle between $\vec{p_1}$ and $\vec{p_2}$, where $\vec{p} = p - O$ and $O$ is the center of coordinates. \\
$\alpha = \arccos{\bigg(\frac{\vec{p_1}\cdot\vec{p_2}}{\|\vec{p_1}\|\|\vec{p_2}\|}\bigg)}$\\
\medskip

\textbf{(b)} Give a formula to detemine the rotation axis \textbf{v}.\\
Rotation axis \textbf{v} is perpendicular to $\vec{p_1}$ and $\vec{p_2}$ and goes through the coordinates center. Therefore, it is the normalized cross-product of the two vectors.\\
$\mathbf{v} = \frac{\vec{p_1} \times \vec{p_2}}{\|\vec{p_1} \times \vec{p_2}\|}$.\\
\medskip

\textbf{(c)} Write down the quaternion \textbf{q} which performs the rotation with angle $\alpha$ around \textbf{v}.\\
$\mathbf{q} = (\cos(\alpha/2), \mathbf{v_x} \sin(\alpha/2), \mathbf{v_y} \sin(\alpha/2), \mathbf{v_z} \sin(\alpha/2))$.\\ 
\medskip

\textbf{(d)} Write down the relationship between $p_1$ and $p_2$ using quaternion multiplication. \\
$p_1 \mapsto q_{p1} = (0, p_1) = x_1i + y_1j + z_1k$. \\
$p_2 \mapsto q_{p2} = (0, p_2) = x_2i + y_2j + z_2k$. \\
$q_{p2} = \mathbf{q} q_{p1} \mathbf{q^{-1}}$.

\section*{Exercise 2}
Composite transformation can be created as a multiplication of every transformation matrix in the corresponding order.\\
\medskip

\textbf{(a)} Derive a matrix $M_1$ which first rotates the point $\alpha$ degrees ($\alpha$ is given in radians) around the axis $\begin{pmatrix}0 & 0 & 1\end{pmatrix}$ and then performs a translation with an offset of $\begin{pmatrix}t_1 & t_2 & t_3\end{pmatrix}^T$. \\
$M_1 = \begin{pmatrix}\cos{\alpha} & -\sin{\alpha} & 0 & 0\\ \sin{\alpha} & \cos{\alpha} & 0 & 0\\ 0 & 0 & 1 & 0 \\ 0 & 0 & 0 & 1 \end{pmatrix} 
\begin{pmatrix}1 & 0 & 0 & t_1\\ 0 & 1 & 0 & t_2\\ 0 & 0 & 1 & t_3\\ 0 & 0 & 0 & 1\end{pmatrix} = \\ =
\begin{pmatrix}\cos{\alpha} & -\sin{\alpha} & 0 & t_1\cos{\alpha} - t_2\sin{\alpha}\\ \sin{\alpha} & \cos{\alpha} & 0 & t_1\sin{\alpha} + t_2\cos{\alpha}\\ 0 & 0 & 1 & t_3 \\ 0 & 0 & 0 & 1 \end{pmatrix}$\\ 
\medskip

\textbf{(b)} Derive a matrix $M_1$ which performs a translation with an offset of $\begin{pmatrix}t_1 & t_2 & t_3\end{pmatrix}^T$ and then rotates the point $\alpha$ degrees ($\alpha$ is given in radians) around the axis $\begin{pmatrix}0 & 0 & 1\end{pmatrix}$. \\
$M_2 = \begin{pmatrix}1 & 0 & 0 & t_1\\ 0 & 1 & 0 & t_2\\ 0 & 0 & 1 & t_3\\ 0 & 0 & 0 & 1\end{pmatrix}
\begin{pmatrix}\cos{\alpha} & -\sin{\alpha} & 0 & 0\\ \sin{\alpha} & \cos{\alpha} & 0 & 0\\ 0 & 0 & 1 & 0 \\ 0 & 0 & 0 & 1 \end{pmatrix} = \\ =
\begin{pmatrix}\cos{\alpha} & -\sin{\alpha} & 0 & t_1\\ \sin{\alpha} & \cos{\alpha} & 0 & t_2\\ 0 & 0 & 1 & t_3 \\ 0 & 0 & 0 & 1 \end{pmatrix}$ \\
\medskip

\textbf{(c)} In which way affects the order of operations the respective final transformation matrix in this case? \\
The importance of operations order can be shown on this picture:\\
\begin{figure}[h]
\includegraphics[scale=0.4]{img1}
\centering
\caption{Visualistion of transformations in different order}
\end{figure}

The order of transformations affects the final matrix, because matrix multiplication is not commutative. That is why in the second case(rotation first) the final matrix is just a composition of two (everything gets multiplied by 1),
and in the first case the final matrix gets a bit more complicated and the translation part of the matrix changes its $x$ and $y$ values. If for example the rotation was around $y$ axis, then $x-$ and $z-$translations would be affected.  

\end{document}
